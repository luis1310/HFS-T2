\chapter{Marco Teórico y Estado del Arte}

Este capítulo establece los fundamentos teóricos necesarios para comprender el problema de optimización abordado en este proyecto, seguido de una revisión del estado actual de las investigaciones en el campo. La primera sección presenta el marco teórico que sustenta el enfoque multiobjetivo implementado, mientras que la segunda sección analiza los avances y aplicaciones más relevantes en la literatura reciente.

\section{Marco Teórico}

Esta sección presenta los fundamentos teóricos que sustentan el desarrollo de este proyecto de tesis, abarcando desde los conceptos de optimización multiobjetivo hasta las particularidades de los problemas de programación en sistemas de manufactura y la importancia de la eficiencia energética.

\subsection{Optimización Multiobjetivo}

La optimización multiobjetivo es un campo de las matemáticas aplicadas que aborda problemas donde múltiples funciones objetivo deben ser optimizadas simultáneamente. A diferencia de los problemas de optimización monoobjetivo, donde existe una única solución óptima, los problemas multiobjetivo generalmente presentan un conjunto de soluciones óptimas conocidas como soluciones Pareto-óptimas \cite{ParetoPFL2024}.

\subsubsection{Dominancia de Pareto y Frente de Pareto}

El concepto de dominancia de Pareto es fundamental en optimización multiobjetivo. Una solución $\mathbf{x}_1$ domina en el sentido de Pareto a otra solución $\mathbf{x}_2$ si $\mathbf{x}_1$ es al menos tan buena como $\mathbf{x}_2$ en todos los objetivos, y estrictamente mejor en al menos uno. El conjunto de soluciones no dominadas forma el frente de Pareto, que representa el espacio de compromiso entre objetivos conflictivos \cite{ParetoSmallData2023}.

Investigaciones recientes han demostrado que el aprendizaje del frente de Pareto (Pareto Front Learning, PFL) mediante hiperrredes permite aproximar eficientemente el frente en problemas multiobjetivo complejos \cite{ParetoPFL2024}. Estudios sobre optimización bajo incertidumbre han mostrado que la cuantificación de incertidumbre aleatoria mejora significativamente la creación del frente de Pareto y la toma de decisiones en procesos con funciones no lineales \cite{ParetoUncertainty2025}.

\subsection{Algoritmos Evolutivos y NSGA-II}

Los algoritmos evolutivos (AE) son metaheurísticas inspiradas en los principios de la evolución biológica, que utilizan mecanismos de selección, cruce y mutación para explorar el espacio de soluciones de problemas de optimización. En el contexto multiobjetivo, el algoritmo NSGA-II (Non-dominated Sorting Genetic Algorithm II) se ha establecido como uno de los métodos más ampliamente adoptados debido a su simplicidad operativa, eficiencia computacional y requisitos mínimos de parámetros \cite{NSGAII2020Workshop}.

\subsubsection{Principios Fundamentales del NSGA-II}

El algoritmo NSGA-II aborda tres dificultades clave de los algoritmos evolutivos multiobjetivo: complejidad computacional, ausencia de elitismo, y la necesidad de especificar parámetros de compartición. El algoritmo utiliza un enfoque de ordenamiento no dominado rápido con complejidad computacional $O(MN^2)$, donde $M$ es el número de objetivos y $N$ es el tamaño de la población \cite{NSGAIIImproved2025}.

Los mecanismos principales del NSGA-II incluyen: (1) clasificación no dominada para identificar los frentes de Pareto en la población, (2) distancia de aglomeración (crowding distance) que reemplaza el radio de compartición y reduce la complejidad computacional, y (3) elitismo mediante la preservación de las mejores soluciones entre generaciones \cite{NSGAII2020Workshop}.

\subsubsection{Mejoras Recientes del NSGA-II}

Investigaciones recientes han propuesto mejoras al algoritmo NSGA-II para problemas específicos. Un estudio de 2024 presentó un operador de cruce híbrido simulado binario y aritmético mejorado (SBAX) aplicado al NSGA-II para problemas bi-objetivo de operación de reservorios \cite{NSGAIIReservoir2024}. Otro trabajo de 2025 introdujo mecanismos adaptativos que ajustan dinámicamente las tasas de cruce y mutación basándose en el progreso generacional, mejorando el rendimiento en problemas de optimización de antenas \cite{NSGAIIImproved2025}.

Comparaciones sistemáticas entre NSGA-II y NSGA-III han confirmado que NSGA-II sigue siendo altamente efectivo para problemas con dos o tres objetivos \cite{NSGAIIvsIII2020}.

\subsection{Problemas de Programación de Producción}

Los problemas de programación de producción en sistemas de manufactura buscan optimizar la asignación de tareas a máquinas y la secuenciación de operaciones. El problema de Hybrid Flow Shop (HFS) representa una generalización del flow shop clásico donde máquinas paralelas están disponibles para realizar la misma operación en cada etapa \cite{HFSRL2025}.

\subsubsection{Características del Hybrid Flow Shop}

Un sistema HFS se caracteriza por tener múltiples etapas de procesamiento, donde cada etapa puede contener una o más máquinas idénticas operando en paralelo. Los pedidos deben procesarse secuencialmente a través de todas las etapas, pero pueden asignarse a cualquier máquina disponible dentro de cada etapa. Esta estructura refleja configuraciones industriales reales en sectores como manufactura de semiconductores, industria textil y procesamiento de alimentos \cite{HFSSemiconductor2025}.

Revisiones sistemáticas recientes han documentado el creciente interés industrial en métodos y algoritmos de optimización para problemas HFS multiobjetivo \cite{HFSSystematicReview2022}. Estudios de 2025 han explorado el uso de aprendizaje por refuerzo para programación HFS, mostrando resultados prometedores en problemas multi-escenario \cite{HFSRL2025}.

\subsubsection{Restricciones y Complejidad}

Los problemas HFS reales incluyen restricciones como tiempos de preparación dependientes de la secuencia (sequence-dependent setup times), restricciones de elegibilidad de fábrica, y sistemas re-entrantes donde los pedidos pueden volver a etapas previas. La complejidad computacional de estos problemas es típicamente NP-difícil, lo que justifica el uso de metaheurísticas como los algoritmos evolutivos \cite{HFSEnergyEfficient2022}.

\subsection{Sostenibilidad Energética en Manufactura}

El aumento de los costos energéticos, las incertidumbres en el suministro de energía, y la crisis de sostenibilidad han intensificado la necesidad de eficiencia energética en la manufactura industrial \cite{EnergyOptimization2025}. Los sistemas de manufactura flexibles en energía (Energy Flexible Manufacturing Systems) buscan apoyar a las empresas en la optimización de eficiencia energética, costos y emisiones de carbono mediante el consumo flexible de energía renovable \cite{EnergyFlexible2025}.

\subsubsection{Programación Consciente de Energía}

La programación consciente de energía (energy-aware scheduling) ha emergido como un área de investigación muy activa, con más de 500 artículos publicados en la última década \cite{EnergySchedulingReview2025}. Esta área aborda la optimización simultánea de objetivos productivos (como el makespan) y objetivos de sostenibilidad (como el consumo energético), reconociendo que estos objetivos son frecuentemente conflictivos.

Investigaciones recientes proponen marcos de optimización multiobjetivo basados en simulación combinados con descubrimiento de conocimiento, que identifican sistemáticamente pérdidas de energía y productividad, extrayendo reglas accionables para guiar la toma de decisiones \cite{EnergyOptimization2025}. Debido a los crecientes requisitos de reducción de emisiones y precios energéticos más altos, las empresas manufactureras han comenzado a prestar atención a la eficiencia energética de las cadenas de suministro \cite{EnergySupplyChain2025}.

\section{Estado del Arte}

Esta sección proporciona un conjunto de antecedentes sobre los avances y enfoques más relevantes en la aplicación de AE, en particular los AG, a la optimización de procesos productivos. Este análisis es fundamental para entender el estado actual de las investigaciones y prácticas en esta área, así como para identificar las oportunidades y limitaciones de las soluciones existentes.

\section{Reducción del Tiempo de Terminación en la Programación de la Producción de una Línea de Flujo Híbrida Flexible (HFS)}
Lopez, et al. \textcite{Lop_et_al_2015} presenta un modelo de programación de la producción utilizando una metaheurística para reducir el tiempo de finalización del makespan en una empresa textil. A través de la codificación de un algoritmo genético simple, se desarrolló una metodología para gestionar la producción en líneas de flujo híbridas flexibles. El algoritmo ofrece resultados de buena calidad con tiempos de ejecución razonables y una variación mínima en el makespan (2\%). Los autores concluyen que el algoritmo genético gestiona eficazmente la producción al reducir el makespan.

\section{Two hybrid flow shop scheduling lines with assembly stage and compatibility constraints}
En su artículo Muñoz, et al. \textcite{Mun_etal_2024} abordan el problema de programación en líneas de producción híbridas, específicamente en el contexto de la fabricación de automóviles. Los principales objetivos del modelo de programación propuesto son minimizar el makespan y optimizar el uso de los recursos disponibles en las líneas de producción. Los autores proponen un modelo de programación lineal entera mixta (MILP) que se complementa con un enfoque matheurístico, ya que este enfoque combina métodos exactos con estrategias metaheurísticas. Los autores utilizan un algoritmo matheurístico de planificación inversa, estableciendo un tiempo de finalización para los trabajos y trabaja hacia atrás para definir los pasos necesarios para cumplir con el tiempo de ensamblaje planificado. Además, se implementan dos metaheurísticas: un procedimiento de búsqueda adaptativa aleatoria codicioso (GRASP) y un algoritmo genético de clave aleatoria sesgada (BRKGA). Los autores concluyen que la combinación de un modelo MILP con un enfoque matheurístico y el uso de metaheurísticas puede ser altamente efectiva para resolver problemas complejos de programación en líneas de producción híbridas

\section{A two-stage hybrid flow-shop formulation for sterilization processes in hospitals}
Kraul \textcite{KRAUL2025125624} se centra en la optimización de los procesos de esterilización en hospitales, un área crítica y costosa en la gestión de dispositivos médicos, con el objetivo de reducir el tiempo que estos dispositivos pasan en el departamento de suministro estéril central (CSSD). Para lograr esto, se desarrolló e implementó un algoritmo basado en reglas de despacho dentro de una formulación de flujo híbrido de dos etapas. La metodología incluyó la comparación de diferentes algoritmos de programación con un modelo de programación entera mixta (MIP) en instancias pequeñas de 25 trabajos, seguido de la evaluación de instancias más grandes utilizando datos reales, donde el número de trabajos variaba entre 70 y 437. Los resultados mostraron que los algoritmos heurísticos, especialmente el algoritmo genético, superaron al MIP en términos de tiempo de solución y brechas promedio, con el algoritmo genético logrando una brecha del 9.74\% en solo 9.81 segundos, en contraste con el MIP, que tuvo una brecha del 14.76\% y un tiempo de 600 segundos. Estas conclusiones subrayan la efectividad de las heurísticas para abordar problemas de programación en entornos hospitalarios, sugiriendo que una programación eficiente de las máquinas puede llevar a ahorros significativos y una mejor utilización de los recursos, mejorando así la eficiencia operativa y la calidad del servicio en el ámbito de la salud.

\section{Multi-objective genetic algorithm for energy-efficient hybrid flow shop scheduling with lot streaming}
Chen, et al. \textcite{Chen2020813} en su estudio aborda el problema de la programación en un entorno de Hybrid Flow Shop (HFS), centrándose en la minimización simultánea del tiempo de finalización (makespan) y el consumo de energía, en un contexto donde la sostenibilidad ambiental es crucial. Para ello, se propone un modelo de programación entera mixta de múltiples objetivos, que permite optimizar la secuenciación de trabajos en máquinas no relacionadas, considerando factores como el tamaño de los sublotes, tiempos de configuración dependientes de la secuencia y fechas de liberación. La metodología empleada incluye un algoritmo genético multiobjetivo, que busca equilibrar la eficiencia de producción y la reducción de la huella de carbono. Los resultados muestran que la implementación de este enfoque no solo mejora el rendimiento en términos de tiempo de finalización, sino que también reduce significativamente el consumo de energía en comparación con métodos tradicionales. Las conclusiones destacan la viabilidad de integrar criterios de sostenibilidad en la programación de operaciones, sugiriendo que la optimización del consumo energético puede lograrse sin comprometer la eficiencia productiva, lo que representa un avance importante hacia prácticas de manufactura más responsables y sostenibles.

\section{An efficient genetic algorithm for hybrid flow shop scheduling with multiprocessor task problems}
Engin, et al. \textcite{Engin20113056} en su estudio se centraron en el desarrollo de un algoritmo genético eficiente para abordar el problema de programación de tareas en un entorno de taller híbrido multiprocesador, con el objetivo de minimizar el tiempo total de finalización (makespan). La metodología empleada incluye la parametrización del algoritmo genético mediante un diseño de experimentos que evalúa múltiples parámetros de control, como la población inicial, métodos de selección, métodos de cruce y mutación, así como sus respectivas tasas. Se utilizó un diseño factorial completo para investigar cómo estos parámetros afectan los resultados del algoritmo. Los resultados obtenidos demostraron que la selección adecuada de los parámetros de control tiene un impacto significativo en la eficiencia del algoritmo, permitiendo una mejora notable en la programación de tareas en comparación con métodos tradicionales. Los autores concluyen que su estudio no solo optimiza el tiempo de finalización, sino que también ofrece un enfoque sistemático para la selección de parámetros.

\section{Reinforcement Learning-Based Multi-Objective of Two-Stage Blocking Hybrid Flow Shop Scheduling Problem}
Xu, et al. \textcite{Xu2024} en su artículo investiga un modelo de programación de múltiples objetivos en un entorno de HFS bloqueado de dos etapas, con el objetivo de minimizar tanto el tiempo total de finalización (makespan) como el consumo total de energía, considerando los tiempos de transporte entre las etapas de procesamiento. Para abordar este problema, se desarrolló un algoritmo de Q-learning adaptativo que incorpora una estrategia de selección de objetivos basada en pruebas t, permitiendo evaluar la confianza en las funciones objetivo bajo el estado actual de trabajos y máquinas. La metodología incluye la formulación de un modelo matemático que considera la congestión de máquinas y la falta de buffers entre etapas, así como la implementación de características del estado basadas en información en tiempo real sobre trabajos y colas de procesamiento. Los resultados de simulaciones realizadas en diferentes escalas experimentales muestran que el algoritmo propuesto logra soluciones óptimas en el 92\%, 83.3\% y 91.7\% de los casos de prueba, superando otros métodos como reglas de programación simples y el algoritmo NSGA-II. Las conclusiones destacan que el enfoque de Q-learning adaptativo no solo mejora la utilización de recursos y reduce el impacto del tiempo de bloqueo y transporte en el tiempo de finalización, sino que también contribuye a un modelo de fabricación más sostenible al disminuir el consumo de energía en los procesos industriales.

\section{Enfoque Multiobjetivo en Programación de Líneas de Ensamblaje}
La literatura revisada muestra una tendencia creciente hacia el enfoque multiobjetivo en problemas de programación. Chen, et al. \textcite{Chen2020813} y Xu, et al. \textcite{Xu2024} demuestran la importancia de considerar simultáneamente el tiempo de finalización y el consumo energético en sistemas de manufactura. Esta evolución refleja la necesidad de los tomadores de decisión de considerar múltiples criterios relevantes en problemas industriales reales.

En el contexto de líneas de ensamblaje, el enfoque multiobjetivo permite explorar el espacio de compromiso entre objetivos conflictivos como productividad, equidad de carga y sostenibilidad, proporcionando al decisor un conjunto de alternativas óptimas según sus prioridades operativas.

Habiendo revisado el estado del arte y identificado las tendencias y gaps en la literatura, el siguiente capítulo presenta la metodología desarrollada para abordar el problema de optimización multiobjetivo en líneas de ensamblaje, incluyendo la formulación matemática, la implementación de los algoritmos evolutivos y las estrategias de optimización.
