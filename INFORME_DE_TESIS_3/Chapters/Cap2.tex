\chapter{Marco teórico y estado del arte}

Este capítulo establece las bases teóricas para entender el problema de optimización tratado en el proyecto, además de revisar el estado actual de la investigación en este campo. La primera sección desarrolla el marco teórico que sustenta el enfoque multiobjetivo implementado. La segunda sección analiza los avances y aplicaciones relevantes que se encuentran en la literatura reciente.

\section{Marco teórico}

Los fundamentos teóricos que dan soporte a este trabajo de tesis se desarrollan en esta sección. Se abordan conceptos de optimización multiobjetivo, características particulares de problemas de programación en sistemas de manufactura, y la importancia de la eficiencia energética.

\subsection{Optimización multiobjetivo}

La optimización multiobjetivo es un campo de las matemáticas aplicadas enfocado en problemas donde se deben optimizar múltiples funciones objetivo simultáneamente. Cuando hay un solo objetivo, existe una única solución óptima. En cambio, cuando hay múltiples objetivos, típicamente aparece un conjunto de soluciones óptimas conocidas como soluciones Pareto-óptimas \cite{ParetoPFL2024}.

\subsubsection{Dominancia de Pareto y frente de Pareto}

El concepto de dominancia de Pareto es fundamental en optimización multiobjetivo. Se dice que una solución $\mathbf{x}_1$ domina a otra solución $\mathbf{x}_2$ en el sentido de Pareto cuando $\mathbf{x}_1$ es al menos tan buena como $\mathbf{x}_2$ en todos los objetivos y estrictamente mejor en al menos uno. El conjunto de soluciones no dominadas forma el frente de Pareto, representando el espacio de compromiso entre objetivos conflictivos \cite{ParetoSmallData2023}.

Investigaciones recientes han demostrado que el aprendizaje del frente de Pareto (Pareto Front Learning, PFL) mediante hiperrredes permite aproximar eficientemente el frente en problemas multiobjetivo complejos \cite{ParetoPFL2024}. Estudios sobre optimización bajo incertidumbre han mostrado que la cuantificación de incertidumbre aleatoria mejora significativamente la creación del frente de Pareto y la toma de decisiones en procesos con funciones no lineales \cite{ParetoUncertainty2025}.

\subsection{Algoritmos evolutivos y NSGA-II}

Los algoritmos evolutivos (AE) son metaheurísticas inspiradas en los principios de la evolución biológica, que utilizan mecanismos de selección, cruce y mutación para explorar el espacio de soluciones de problemas de optimización. En el contexto multiobjetivo, el algoritmo NSGA-II (Non-dominated Sorting Genetic Algorithm II) se ha establecido como uno de los métodos más ampliamente adoptados debido a su simplicidad operativa, eficiencia computacional y requisitos mínimos de parámetros \cite{NSGAII2020Workshop}.

\subsubsection{Principios fundamentales del NSGA-II}

El algoritmo NSGA-II aborda tres dificultades clave de los algoritmos evolutivos multiobjetivo: complejidad computacional, ausencia de elitismo, y la necesidad de especificar parámetros de compartición. El algoritmo utiliza un enfoque de ordenamiento no dominado rápido con complejidad computacional $O(MN^2)$, donde $M$ es el número de objetivos y $N$ es el tamaño de la población \cite{NSGAIIImproved2025}.

Los mecanismos principales del NSGA-II incluyen: (1) clasificación no dominada para identificar los frentes de Pareto en la población, (2) distancia de aglomeración (crowding distance) que reemplaza el radio de compartición y reduce la complejidad computacional, y (3) elitismo mediante la preservación de las mejores soluciones entre generaciones \cite{NSGAII2020Workshop}.

\subsubsection{Mejoras recientes del NSGA-II}

Investigaciones recientes han propuesto mejoras al algoritmo NSGA-II para problemas específicos. Un estudio de 2024 presentó un operador de cruce híbrido simulado binario y aritmético mejorado (SBAX) aplicado al NSGA-II para problemas bi-objetivo de operación de reservorios \cite{NSGAIIReservoir2024}. Otro trabajo de 2025 introdujo mecanismos adaptativos que ajustan dinámicamente las tasas de cruce y mutación basándose en el progreso generacional, mejorando el rendimiento en problemas de optimización de antenas \cite{NSGAIIImproved2025}.

Comparaciones sistemáticas entre NSGA-II y NSGA-III han confirmado que NSGA-II sigue siendo altamente efectivo para problemas con dos o tres objetivos \cite{NSGAIIvsIII2020}.

%%%%%%%%%%%%%%%%%%%%%%%%%%%%%%%%%%%%%%%%%%%%%%%%%%%%%%%%%%%%%%%%%%%%%%%%%%
%%% SUBSECCIÓN AGREGADA PARA REVISIÓN: OVERHEAD COMPUTACIONAL
%%%%%%%%%%%%%%%%%%%%%%%%%%%%%%%%%%%%%%%%%%%%%%%%%%%%%%%%%%%%%%%%%%%%%%%%%%

\subsubsection{Eficiencia Computacional y Overhead en Algoritmos Evolutivos}

La eficiencia computacional es un aspecto fundamental en el diseño y aplicación de algoritmos evolutivos, especialmente cuando se abordan problemas de optimización multiobjetivo complejos que requieren evaluaciones extensivas del espacio de soluciones. El concepto de \textit{overhead computacional} (sobrecarga computacional) se refiere al tiempo de procesamiento adicional que se introduce cuando se incorporan componentes o mecanismos específicos a un algoritmo base \cite{MetaheuristicSurvey2024}.

El overhead computacional se mide típicamente como el incremento porcentual en el tiempo de ejecución respecto a la versión base del algoritmo. Por ejemplo, si un algoritmo base ejecuta en $T_{\text{base}}$ segundos y la versión mejorada ejecuta en $T_{\text{mejorado}}$ segundos, el overhead se calcula como:
\begin{equation}
\text{Overhead} = \frac{T_{\text{mejorado}} - T_{\text{base}}}{T_{\text{base}}} \times 100\%
\end{equation}

En el contexto de algoritmos evolutivos multiobjetivo, el overhead puede originarse de diversos componentes, como mecanismos de búsqueda local adicionales, operadores de refinamiento, estrategias adaptativas de parámetros, o procedimientos de filtrado de soluciones \cite{MemeticAlgorithmsGuide2025}. El desafío principal consiste en balancear el overhead introducido con la mejora en la calidad de las soluciones obtenidas \cite{FutureMemeticAlgorithms2025}.

Las investigaciones recientes demuestran que la justificación del overhead depende del contexto de aplicación. En problemas donde la calidad de la solución es crítica, un overhead del 15-20\% puede ser aceptable si se traduce en mejoras significativas en los objetivos de optimización \cite{EnhancedNSGAII2023}. Inversamente, en entornos que requieren respuestas rápidas o procesan grandes volúmenes de instancias, la eficiencia computacional puede priorizar sobre mejoras marginales en calidad \cite{NSGAIIBakery2022}.

La evaluación del overhead es particularmente relevante en algoritmos meméticos, donde la incorporación de búsqueda local puede mejorar la convergencia hacia el óptimo pero incrementa sustancialmente el costo computacional por generación. El análisis del trade-off entre calidad y overhead permite a los investigadores y profesionales seleccionar la configuración más apropiada según los requisitos específicos del problema y las restricciones de tiempo disponibles.

%%%%%%%%%%%%%%%%%%%%%%%%%%%%%%%%%%%%%%%%%%%%%%%%%%%%%%%%%%%%%%%%%%%%%%%%%%
%%% FIN DE SUBSECCIÓN AGREGADA
%%%%%%%%%%%%%%%%%%%%%%%%%%%%%%%%%%%%%%%%%%%%%%%%%%%%%%%%%%%%%%%%%%%%%%%%%%

\subsection{Problemas de Programación de Producción}

En manufactura, programar la producción significa optimizar cómo se asignan tareas a las máquinas y en qué orden se ejecutan las operaciones. El Hybrid Flow Shop (HFS) extiende el concepto de flow shop clásico al incluir varias máquinas trabajando en paralelo dentro de cada etapa para hacer la misma operación \cite{HFSRL2025}.

\subsubsection{Características del Hybrid Flow Shop}

Un sistema HFS tiene varias etapas de procesamiento, y cada etapa puede tener una o más máquinas iguales funcionando simultáneamente. Los pedidos pasan por todas las etapas en secuencia, aunque dentro de cada etapa se pueden asignar a cualquier máquina que esté libre. Esta configuración aparece en industrias reales como semiconductores, textiles y alimentos \cite{HFSSemiconductor2025}.

Revisiones sistemáticas recientes muestran que la industria tiene cada vez más interés en métodos de optimización para HFS multiobjetivo \cite{HFSSystematicReview2022}. Trabajos publicados en 2025 han investigado aprendizaje por refuerzo aplicado a HFS, obteniendo resultados prometedores cuando hay múltiples escenarios \cite{HFSRL2025}.

\subsubsection{Restricciones y complejidad}

En problemas HFS reales hay restricciones complejas: tiempos de setup que varían según la secuencia, limitaciones sobre qué fábricas usar, y sistemas donde los pedidos pueden regresar a etapas anteriores. Estos problemas suelen ser NP-difíciles computacionalmente, por eso tiene sentido usar metaheurísticas como algoritmos evolutivos \cite{HFSEnergyEfficient2022}.

\subsection{Sostenibilidad energética en manufactura}

Los costos energéticos han subido, el suministro es incierto, y hay una crisis de sostenibilidad. Por esto, la eficiencia energética en manufactura se ha vuelto más urgente \cite{EnergyOptimization2025}. Los Energy Flexible Manufacturing Systems ayudan a las empresas a optimizar eficiencia energética, costos y emisiones usando energía renovable de manera flexible \cite{EnergyFlexible2025}.

\subsubsection{Programación consciente de energía}

La programación consciente de energía (energy-aware scheduling) se ha vuelto un área de investigación muy activa; se publicaron más de 500 artículos sobre esto en los últimos diez años \cite{EnergySchedulingReview2025}. Esta área optimiza al mismo tiempo objetivos productivos (makespan) y de sostenibilidad (consumo energético), sabiendo que frecuentemente estos objetivos chocan entre sí.

Estudios recientes usan marcos que combinan simulación con descubrimiento de conocimiento. Estos marcos encuentran sistemáticamente dónde se pierde energía y productividad, y generan reglas prácticas para decisiones \cite{EnergyOptimization2025}. Como hay que reducir más las emisiones y la energía cuesta más, las empresas manufactureras están prestando atención a qué tan eficientes energéticamente son sus cadenas de suministro \cite{EnergySupplyChain2025}.

\section{Estado del arte}

Esta sección proporciona un conjunto de antecedentes sobre los avances y enfoques más relevantes en la aplicación de AE, en particular los AG, a la optimización de procesos productivos. Este análisis es fundamental para entender el estado actual de las investigaciones y prácticas en esta área, así como para identificar las oportunidades y limitaciones de las soluciones existentes.

\section{Reducción del tiempo de terminación en la programación de la producción de una línea de flujo híbrida flexible (HFS)}
\textcite{Lop_et_al_2015} presenta un modelo de programación de la producción utilizando una metaheurística para reducir el tiempo de finalización del makespan en una empresa textil. A través de la codificación de un algoritmo genético simple, se desarrolló una metodología para gestionar la producción en líneas de flujo híbridas flexibles. El algoritmo ofrece resultados de buena calidad con tiempos de ejecución razonables y una variación mínima en el makespan (2\%). Los autores concluyen que el algoritmo genético gestiona eficazmente la producción al reducir el makespan.

\section{Two hybrid flow shop scheduling lines with assembly stage and compatibility constraints}
En su artículo \textcite{Mun_etal_2024} abordan el problema de programación en líneas de producción híbridas, específicamente en el contexto de la fabricación de automóviles. Los principales objetivos del modelo de programación propuesto son minimizar el makespan y optimizar el uso de los recursos disponibles en las líneas de producción. Los autores proponen un modelo de programación lineal entera mixta (MILP) que se complementa con un enfoque matheurístico, ya que este enfoque combina métodos exactos con estrategias metaheurísticas. Los autores utilizan un algoritmo matheurístico de planificación inversa, estableciendo un tiempo de finalización para los trabajos y trabaja hacia atrás para definir los pasos necesarios para cumplir con el tiempo de ensamblaje planificado. Además, se implementan dos metaheurísticas: un procedimiento de búsqueda adaptativa aleatoria codicioso (GRASP) y un algoritmo genético de clave aleatoria sesgada (BRKGA). Los autores concluyen que la combinación de un modelo MILP con un enfoque matheurístico y el uso de metaheurísticas puede ser altamente efectiva para resolver problemas complejos de programación en líneas de producción híbridas

\section{A two-stage hybrid flow-shop formulation for sterilization processes in hospitals}
\textcite{KRAUL2025125624} se centra en la optimización de los procesos de esterilización en hospitales, un área crítica y costosa en la gestión de dispositivos médicos, con el objetivo de reducir el tiempo que estos dispositivos pasan en el departamento de suministro estéril central (CSSD). Para lograr esto, se desarrolló e implementó un algoritmo basado en reglas de despacho dentro de una formulación de flujo híbrido de dos etapas. La metodología incluyó la comparación de diferentes algoritmos de programación con un modelo de programación entera mixta (MIP) en instancias pequeñas de 25 trabajos, seguido de la evaluación de instancias más grandes utilizando datos reales, donde el número de trabajos variaba entre 70 y 437. Los resultados mostraron que los algoritmos heurísticos, especialmente el algoritmo genético, superaron al MIP en términos de tiempo de solución y brechas promedio, con el algoritmo genético logrando una brecha del 9.74\% en solo 9.81 segundos, en contraste con el MIP, que tuvo una brecha del 14.76\% y un tiempo de 600 segundos. Estas conclusiones subrayan la efectividad de las heurísticas para abordar problemas de programación en entornos hospitalarios, sugiriendo que una programación eficiente de las máquinas puede llevar a ahorros significativos y una mejor utilización de los recursos, mejorando así la eficiencia operativa y la calidad del servicio en el ámbito de la salud.

\section{Multi-objective genetic algorithm for energy-efficient hybrid flow shop scheduling with lot streaming}
El trabajo de \textcite{Chen2020813} trata la programación en HFS buscando minimizar tanto el makespan como el consumo energético, considerando que la sostenibilidad ambiental es un factor crítico. Los autores proponen un modelo de programación entera mixta multiobjetivo que optimiza la secuencia de trabajos en máquinas no relacionadas, tomando en cuenta tamaño de sublotes, setup times dependientes de secuencia y fechas de liberación. Usan un algoritmo genético multiobjetivo para balancear eficiencia productiva con reducción de huella de carbono. Los resultados indican que este enfoque mejora el makespan y reduce considerablemente el consumo energético respecto a métodos convencionales. Los autores concluyen que es viable integrar sostenibilidad en la programación de operaciones, y que se puede optimizar el consumo energético sin sacrificar eficiencia, lo cual representa un progreso hacia manufactura más responsable.

\section{An efficient genetic algorithm for hybrid flow shop scheduling with multiprocessor task problems}
\textcite{Engin20113056} desarrollaron un algoritmo genético eficiente para programación de tareas en talleres híbridos multiprocesador, buscando minimizar el makespan. Su metodología incluye parametrizar el algoritmo genético usando diseño de experimentos para evaluar parámetros como población inicial, métodos de selección, cruce, mutación y sus tasas. Usaron diseño factorial completo para ver cómo estos parámetros impactan el desempeño del algoritmo. Los resultados muestran que elegir bien los parámetros afecta significativamente la eficiencia del algoritmo, mejorando la programación respecto a métodos tradicionales. Los autores destacan que su trabajo no solo optimiza makespan, sino que también da un enfoque sistemático para seleccionar parámetros.

\section{Reinforcement learning-based multi-objective of two-stage blocking hybrid flow shop scheduling problem}
\textcite{Xu2024} estudian un modelo multiobjetivo para HFS bloqueado de dos etapas. Buscan minimizar makespan y consumo energético, considerando tiempos de transporte. Desarrollaron un algoritmo de Q-learning adaptativo con selección de objetivos basada en pruebas t para evaluar la confianza en las funciones objetivo según el estado actual de trabajos y máquinas. La metodología formula un modelo matemático que considera congestión de máquinas y falta de buffers. También usa información en tiempo real de trabajos y colas para definir características de estado. Simulaciones en distintas escalas muestran que el algoritmo encuentra soluciones óptimas en 92\%, 83.3\% y 91.7\% de los casos, superando reglas simples de programación y NSGA-II. Los autores concluyen que Q-learning adaptativo mejora el uso de recursos, reduce el impacto de bloqueo y transporte en makespan, y contribuye a manufactura más sostenible al bajar el consumo energético.

\section{Enfoque multiobjetivo en programación de líneas de ensamblaje}
La literatura muestra que hay cada vez más interés en enfoques multiobjetivo para problemas de programación. \textcite{Chen2020813} y \textcite{Xu2024} muestran que es importante considerar makespan y consumo energético al mismo tiempo en manufactura. Esto refleja que quienes toman decisiones necesitan considerar varios criterios relevantes en problemas industriales reales.

Para líneas de ensamblaje, el enfoque multiobjetivo permite explorar compromisos entre objetivos en conflicto (productividad, balance de carga, sostenibilidad), dando al decisor varias alternativas óptimas según sus prioridades.

Tras revisar el estado del arte e identificar tendencias y vacíos, el siguiente capítulo presenta la metodología desarrollada. Esta incluye formulación matemática, implementación de algoritmos evolutivos y estrategias de optimización para el problema multiobjetivo en líneas de ensamblaje.
