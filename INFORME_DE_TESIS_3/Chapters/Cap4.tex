\chapter{Resultados y Discusión}

\section{Resultados}

\subsection{Experimentos de Tunning Multiobjetivo}

Se realizó un experimento exhaustivo de tunning de hiperparámetros para el algoritmo NSGA-II memético, evaluando un espacio de búsqueda de 1,890 configuraciones diferentes. Cada configuración se ejecutó con 30 semillas independientes para garantizar robustez estadística, resultando en un total de 56,700 ejecuciones.

\subsubsection{Espacio de Búsqueda Evaluado}

El espacio de búsqueda consideró los siguientes rangos de hiperparámetros:

\begin{itemize}
    \item \textbf{Tamaño de población}: 100, 150, 200 individuos
    \item \textbf{Número de generaciones}: 400, 500, 600 generaciones
    \item \textbf{Probabilidad de cruce}: 0.6, 0.7, 0.8, 0.9, 0.95
    \item \textbf{Probabilidad de mutación}: 0.5, 0.75, 0.1, 0.15, 0.20, 0.25, 0.30
    \item \textbf{Frecuencia de búsqueda local}: cada 5, 10, 20 generaciones
    \item \textbf{Iteraciones de búsqueda local}: 3, 5 iteraciones
\end{itemize}

\subsubsection{Mejor Configuración Encontrada}

La mejor configuración identificada mediante el score balanceado (que considera tanto el score agregado como el tiempo de ejecución) fue:

\begin{itemize}
    \item \textbf{Tamaño de población}: 100 individuos
    \item \textbf{Número de generaciones}: 400 generaciones
    \item \textbf{Probabilidad de cruce}: 0.6
    \item \textbf{Probabilidad de mutación}: 0.75
    \item \textbf{Búsqueda local cada}: 20 generaciones
    \item \textbf{Iteraciones locales}: 5 iteraciones
\end{itemize}

Esta configuración obtuvo un score agregado promedio de 2.5997, con las siguientes métricas promedio:
\begin{itemize}
    \item Makespan: 1556.77 minutos
    \item Balance: 278.65
    \item Energía: 624.73 kWh
    \item Tiempo de ejecución: 27.23 segundos
    \item Tamaño del frente de Pareto: 100 soluciones
\end{itemize}

Esta configuración fue seleccionada considerando un balance entre la calidad de las soluciones (score agregado) y el tiempo de ejecución, proporcionando un rendimiento óptimo con un tiempo de ejecución significativamente menor (27.23 segundos vs. 73.88 segundos) comparado con configuraciones que priorizan únicamente el score agregado, manteniendo una calidad de soluciones comparable.

\subsection{Comparación de Operadores}

Se evaluaron sistemáticamente 6 combinaciones diferentes de operadores de cruce y mutación, cada una ejecutada con 30 semillas independientes. Las combinaciones evaluadas fueron:

\begin{itemize}
    \item Cruce uniforme + Mutación swap
    \item Cruce uniforme + Mutación insert
    \item Cruce uniforme + Mutación invert
    \item Cruce un punto + Mutación swap
    \item Cruce un punto + Mutación insert
    \item Cruce un punto + Mutación invert
\end{itemize}

\subsubsection{Mejor Combinación de Operadores}

La mejor combinación identificada fue \textbf{Cruce Uniforme + Mutación Swap}, con las siguientes métricas promedio (sobre 30 semillas):

\begin{itemize}
    \item Score agregado: 2.6065 ± 0.0086
    \item Makespan: 1559.40 ± 3.76 minutos
    \item Balance: 280.34 ± 2.23
    \item Energía: 624.63 ± 0.21 kWh
    \item Tiempo de ejecución: 25.99 segundos
    \item Tamaño del frente: 100 soluciones
\end{itemize}

Esta combinación demostró ser superior en términos de balance entre los tres objetivos, mostrando una buena convergencia y diversidad en el frente de Pareto.

\subsection{Comparación NSGA-II Estándar vs. Memético}

Se realizó una comparación sistemática entre la versión estándar del NSGA-II y su variante memética, utilizando la mejor configuración de hiperparámetros y operadores identificada. Cada versión se ejecutó con 30 semillas independientes.

\subsubsection{Resultados Comparativos}

Los resultados promedio (sobre 30 semillas) muestran que la versión memética proporciona mejoras moderadas pero consistentes:

\begin{table}[H]
\centering
\caption{Comparación NSGA-II Estándar vs. Memético}
\label{tab:comparacion_memetico}
\begin{tabular}{|l|c|c|}
\hline
\textbf{Métrica} & \textbf{Estándar} & \textbf{Memético} \\
\hline
Makespan (min) & 1549.40 ± 7.20 & 1547.45 ± 5.91 \\
Balance & 284.85 ± 2.27 & 284.31 ± 3.00 \\
Energía (kWh) & 623.72 ± 0.45 & 623.61 ± 0.38 \\
Score agregado & 2.6152 ± 0.0100 & 2.6123 ± 0.0115 \\
Tiempo (s) & 60.81 ± 13.24 & 67.02 ± 17.02 \\
\hline
\end{tabular}
\end{table}

\subsubsection{Análisis de Mejoras}

La versión memética muestra mejoras en todos los objetivos:

\begin{itemize}
    \item \textbf{Makespan}: Mejora del 0.13\% (reducción de ~2 minutos)
    \item \textbf{Balance}: Mejora del 0.19\% (reducción de ~0.5 unidades)
    \item \textbf{Energía}: Mejora del 0.02\% (reducción de ~0.1 kWh)
    \item \textbf{Score agregado}: Mejora del 0.11\%
\end{itemize}

El overhead computacional de la versión memética es del 10.22\%, lo cual es aceptable considerando las mejoras obtenidas en la calidad de las soluciones. La versión memética se recomienda para aplicaciones donde la calidad de las soluciones es prioritaria sobre el tiempo de ejecución.

\subsection{Análisis del Frente de Pareto}

Utilizando la mejor configuración identificada (NSGA-II memético con cruce uniforme y mutación swap), se generó un frente de Pareto final con 79 soluciones únicas después de aplicar el filtro de similitud ($\epsilon = 1\%$).

\subsubsection{Características del Frente}

El frente de Pareto obtenido presenta las siguientes características:

\begin{itemize}
    \item \textbf{Rango de makespan}: 1560.61 minutos (rango mínimo, todas las soluciones convergen a este valor)
    \item \textbf{Rango de balance}: 261.03 - 323.70 (rango de 62.67 unidades)
    \item \textbf{Rango de energía}: 622.40 - 628.46 kWh (rango de 6.06 kWh)
\end{itemize}

Estos rangos muestran que el algoritmo logra explorar efectivamente el espacio de compromiso entre los tres objetivos. Es importante destacar que, debido a la naturaleza multiobjetivo del problema, \textbf{no existe una solución única que optimice simultáneamente los tres objetivos}. En su lugar, el frente de Pareto proporciona un conjunto de soluciones alternativas donde cada una representa un compromiso diferente entre los objetivos.

El decisor puede seleccionar la solución más adecuada según sus preferencias específicas:
\begin{itemize}
    \item Si se prioriza la \textbf{productividad temporal}, todas las soluciones del frente tienen makespan de 1560.61 minutos, por lo que la selección se basa en los otros objetivos.
    \item Si se prioriza la \textbf{equidad operativa}, se puede elegir una solución con balance mínimo (261.03), aunque esto puede aumentar el consumo energético (628.46 kWh).
    \item Si se prioriza la \textbf{sostenibilidad energética}, se puede elegir una solución con consumo energético mínimo (622.40 kWh), aunque esto puede resultar en mayor desbalance de carga (323.70).
    \item Si se busca un \textbf{compromiso equilibrado}, se puede elegir una solución intermedia que balancee balance y energía de manera razonable, manteniendo el makespan constante.
\end{itemize}

Esta flexibilidad es una de las principales ventajas del enfoque multiobjetivo: proporciona al decisor múltiples alternativas bien fundamentadas, permitiendo una toma de decisiones informada según el contexto específico y las prioridades del momento.

La visualización del frente de Pareto en el espacio tridimensional de objetivos se muestra en la Figura \ref{fig:frente_3d}, donde cada punto representa una solución no dominada. La distribución de las soluciones muestra claramente el trade-off entre los tres objetivos: soluciones con makespan menor tienden a tener mayor balance o consumo energético, y viceversa.

\begin{figure}[H]
\centering
\includegraphics[width=0.9\textwidth]{Figures/frente_pareto_memético_3d.png}
\caption{Frente de Pareto en el espacio tridimensional (Makespan vs Balance vs Energía)}
\label{fig:frente_3d}
\end{figure}

Para facilitar el análisis de las relaciones entre pares de objetivos, se generaron proyecciones bidimensionales del frente de Pareto, mostradas en la Figura \ref{fig:frente_2d}. Estas proyecciones permiten visualizar claramente los compromisos entre:
\begin{itemize}
    \item Makespan y Balance de Carga
    \item Makespan y Consumo Energético
    \item Balance de Carga y Consumo Energético
\end{itemize}

\begin{figure}[H]
\centering
\includegraphics[width=1.0\textwidth]{Figures/frente_pareto_memético_2d.png}
\caption{Proyecciones bidimensionales del frente de Pareto}
\label{fig:frente_2d}
\end{figure}

\subsubsection{Evolución del Frente}

La evolución del tamaño del frente de Pareto a lo largo de las 400 generaciones muestra un patrón característico:

\begin{itemize}
    \item \textbf{Generaciones iniciales (0-100)}: El frente crece rápidamente desde valores pequeños (10-20 soluciones) hasta valores elevados. Sin embargo, gracias al filtro de similitud aplicado periódicamente, el tamaño del frente nunca alcanza el máximo teórico de la población (100 soluciones), ya que las soluciones similares son eliminadas antes de que se acumulen.
    \item \textbf{Generaciones intermedias (100-250)}: El frente se mantiene en valores altos pero controlados, con reducciones periódicas cuando se aplica el filtro de similitud (cada 30 generaciones). Estas reducciones eliminan soluciones redundantes, manteniendo solo aquellas que son significativamente diferentes en el espacio objetivo.
    \item \textbf{Generaciones avanzadas (250-400)}: El tamaño del frente se estabiliza alrededor de 60-90 soluciones únicas después de cada filtrado, mostrando que el algoritmo ha convergido hacia un conjunto de soluciones no dominadas bien diversificado. El filtro asegura que estas soluciones representen compromisos genuinamente diferentes entre los tres objetivos.
\end{itemize}

El filtro de similitud ($\epsilon = 1\%$) aplicado periódicamente asegura que el frente final contenga solo soluciones significativamente diferentes, evitando la acumulación de soluciones prácticamente idénticas. Esto es fundamental porque, aunque el algoritmo podría generar hasta 100 soluciones no dominadas, el filtro previene que se alcance este máximo al eliminar soluciones redundantes que no aportan diversidad real al conjunto de alternativas disponibles para el decisor.

La evolución del tamaño del frente de Pareto a lo largo de las generaciones se visualiza en la Figura \ref{fig:evolucion_frente}, donde se puede observar claramente el patrón descrito: crecimiento inicial, estabilización con reducciones periódicas debido al filtro, y convergencia final hacia un conjunto de soluciones únicas bien diversificado.

\begin{figure}[H]
\centering
\includegraphics[width=0.9\textwidth]{Figures/evolucion_frente_memético.png}
\caption{Evolución del tamaño del frente de Pareto a lo largo de 400 generaciones}
\label{fig:evolucion_frente}
\end{figure}

\subsubsection{Análisis de Soluciones Extremas}

Para facilitar la selección de soluciones según diferentes criterios, se identificaron las soluciones extremas del frente de Pareto que optimizan cada objetivo individualmente. La Figura \ref{fig:mejores_soluciones} muestra estas soluciones destacadas junto con sus valores en los tres objetivos, permitiendo al decisor visualizar rápidamente las opciones disponibles.

\begin{figure}[H]
\centering
\includegraphics[width=0.9\textwidth]{Figures/mejores_soluciones_pareto.png}
\caption{Mejores soluciones del frente de Pareto según cada objetivo}
\label{fig:mejores_soluciones}
\end{figure}

Estas visualizaciones facilitan la comprensión del frente de Pareto y ayudan al decisor a identificar rápidamente qué soluciones se alinean mejor con sus objetivos prioritarios, ya sea minimizar el tiempo de producción, equilibrar la carga de trabajo o reducir el consumo energético.

\subsection{Optimizaciones y Tiempos de Ejecución}

Las optimizaciones implementadas (caché de fitness, clasificación adaptativa, búsqueda local selectiva) lograron reducir significativamente el tiempo de ejecución:

\begin{itemize}
    \item \textbf{Tiempo promedio por ejecución}: 27.23 segundos (con la mejor configuración: 100 individuos, 400 generaciones)
    \item \textbf{Reducción respecto a versión inicial}: Aproximadamente 60\% de reducción (de ~60 segundos a ~27 segundos)
    \item \textbf{Reducción respecto a configuración de mayor calidad}: La configuración seleccionada (100, 400) reduce el tiempo de ejecución en 63\% comparado con configuraciones de mayor calidad pero más lentas (150, 600: 73.88 segundos), manteniendo una calidad de soluciones comparable
    \item \textbf{Eficiencia del caché}: El caché de fitness evita aproximadamente el 30-40\% de recálculos en generaciones avanzadas
\end{itemize}

La paralelización de experimentos permitió ejecutar los 56,700 experimentos de tunning en un tiempo razonable, distribuyendo la carga computacional entre múltiples núcleos del procesador.

\subsection{Estudio de Ablación}

Para validar científicamente que cada componente del algoritmo es necesario y cuantificar su impacto individual, se realizó un estudio de ablación sistemático. Este estudio evaluó 11 variantes diferentes del algoritmo, cada una ejecutada con 30 semillas independientes, resultando en un total de 330 ejecuciones.

\subsubsection{Variantes Evaluadas}

El estudio de ablación evaluó las siguientes variantes:

\begin{itemize}
    \item \textbf{NSGA2 completo}: Con filtro epsilon, cruce uniforme y mutación swap
    \item \textbf{NSGA2 sin filtro}: Sin filtro epsilon (base NSGA2)
    \item \textbf{NSGA2 con diferentes operadores}: Cruce one-point, mutación insert, mutación invert
    \item \textbf{Memético completo}: Con filtro epsilon, búsqueda local, cache y optimizaciones
    \item \textbf{Memético sin filtro}: Sin filtro epsilon, solo búsqueda local
    \item \textbf{Memético sin búsqueda local}: Sin búsqueda local, solo cache y optimizaciones
    \item \textbf{Memético sin filtro ni búsqueda}: Base memético con solo cache y optimizaciones
    \item \textbf{Memético con diferentes operadores}: Cruce one-point, mutación insert
\end{itemize}

\subsubsection{Impacto de Componentes}

Los resultados del estudio de ablación permiten cuantificar el impacto de cada componente principal:

\begin{table}[H]
\centering
\caption{Impacto de Componentes en el Algoritmo Memético}
\label{tab:impacto_ablacion}
\begin{tabular}{|l|c|c|}
\hline
\textbf{Componente} & \textbf{Impacto Tiempo} & \textbf{Impacto Score} \\
\hline
Filtro Epsilon & +2.33\% & -0.32\% (mejora) \\
Búsqueda Local & +3.03\% & -0.02\% (mejora marginal) \\
\hline
\end{tabular}
\end{table}

\paragraph{Impacto del Filtro Epsilon}

El filtro epsilon de similitud muestra un impacto positivo en la calidad de las soluciones con un costo computacional mínimo:
\begin{itemize}
    \item \textbf{En NSGA-II}: Sin filtro aumenta el tiempo en 2.55\% pero empeora el score en 0.14\%
    \item \textbf{En Memético}: Sin filtro aumenta el tiempo en 2.33\% pero empeora el score en 0.32\%
    \item \textbf{Conclusión}: El filtro epsilon mejora significativamente la calidad (reduce soluciones redundantes) con un overhead computacional mínimo, justificando su inclusión en el algoritmo final.
\end{itemize}

\paragraph{Impacto de la Búsqueda Local}

La búsqueda local en la versión memética muestra un impacto marginal en la calidad pero con un overhead computacional pequeño:
\begin{itemize}
    \item Sin búsqueda local aumenta el tiempo en 3.03\% pero el score empeora solo en 0.02\%
    \item \textbf{Conclusión}: La búsqueda local tiene un impacto mínimo pero positivo en la calidad, con un overhead aceptable. Su inclusión se justifica para robustez y refinamiento de soluciones.
\end{itemize}

\paragraph{Impacto de Operadores Genéticos}

La comparación de operadores genéticos en el contexto de ablación confirma los resultados de la comparación sistemática previa:
\begin{itemize}
    \item \textbf{Cruce uniforme + Mutación swap}: Proporciona el mejor balance tiempo/calidad (21.08s, score 2.6034)
    \item \textbf{NSGA2 + Mutación invert}: Mejor score absoluto (2.6016) pero significativamente más lento (29.32s, +39\% tiempo)
    \item \textbf{Conclusión}: La combinación uniforme + swap seleccionada es óptima para el balance deseado entre calidad y tiempo de ejecución.
\end{itemize}

\subsubsection{Validación del Diseño}

Los resultados del estudio de ablación validan científicamente el diseño del algoritmo:
\begin{itemize}
    \item \textbf{Filtro epsilon}: Componente crítico que mejora la calidad con costo mínimo
    \item \textbf{Búsqueda local}: Componente útil que aporta robustez y refinamiento
    \item \textbf{Operadores seleccionados}: La combinación uniforme + swap es óptima para el problema
    \item \textbf{Configuración final}: Todos los componentes contribuyen positivamente al rendimiento
\end{itemize}

Esta validación científica mediante ablación proporciona evidencia empírica sólida de que cada componente incluido en el algoritmo final es necesario y contribuye al rendimiento global del sistema.

\section{Discusión}

\subsection{Interpretación de Resultados}

Los resultados obtenidos demuestran la efectividad del enfoque multiobjetivo para el problema de programación en líneas de ensamblaje. El algoritmo NSGA-II memético logra encontrar un conjunto diverso de soluciones que representan diferentes compromisos entre los tres objetivos considerados.

La mejora moderada pero consistente de la versión memética sobre la estándar (0.11\% en score agregado) justifica el overhead computacional del 10.22\%, especialmente en aplicaciones donde la calidad de las soluciones es crítica. La búsqueda local permite refinar las soluciones del frente de Pareto, mejorando la convergencia hacia soluciones de mayor calidad.

\subsection{Comparación con Trabajos Relacionados}

Los resultados obtenidos contrastan favorablemente con trabajos previos en el área:

\begin{itemize}
    \item \textbf{Robustez}: La desviación estándar del score agregado (0.0086-0.0115) es significativamente menor que la variación del 0.217\% reportada en trabajos monoobjetivo previos \cite{Lop_et_al_2015}, demostrando mayor consistencia en los resultados.
    
    \item \textbf{Complejidad}: Aunque el tiempo de ejecución (25-30 segundos) es mayor que los 9.81 segundos reportados por Kraul \cite{KRAUL2025125624} para problemas de 2 etapas, el presente trabajo aborda un problema más complejo (5 etapas, 11 máquinas) y proporciona soluciones multiobjetivo, lo cual es más valioso para la toma de decisiones.
    
    \item \textbf{Enfoque multiobjetivo}: A diferencia de trabajos que se enfocan únicamente en minimizar makespan \cite{Mun_etal_2024}, el presente trabajo considera simultáneamente productividad, equidad operativa y sostenibilidad energética, proporcionando una visión más completa del problema.
\end{itemize}

\subsection{Limitaciones y Consideraciones}

Es importante reconocer las siguientes limitaciones del trabajo:

\begin{itemize}
    \item \textbf{Escalabilidad}: Los experimentos se realizaron con 40 pedidos. La escalabilidad a instancias más grandes (100+ pedidos) requiere evaluación adicional.
    
    \item \textbf{Tiempo de ejecución}: Aunque se logró una reducción significativa, el tiempo de ejecución (25-30 segundos) puede ser limitante para aplicaciones en tiempo real con instancias muy grandes.
    
    \item \textbf{Valores de referencia}: El score agregado utiliza valores de referencia conservadores. Aunque esto no afecta las comparaciones relativas, los valores absolutos deben interpretarse con cautela.
\end{itemize}

\subsection{Contribuciones Principales}

Este trabajo contribuye al área de optimización multiobjetivo en líneas de ensamblaje mediante:

\begin{enumerate}
    \item La formulación multiobjetivo del problema HFS considerando makespan, balance de carga y consumo energético simultáneamente.
    
    \item La implementación y optimización del algoritmo NSGA-II memético con operadores especializados para problemas de programación en líneas de ensamblaje.
    
    \item Un proceso sistemático de tunning multiobjetivo que evalúa exhaustivamente el espacio de hiperparámetros considerando múltiples métricas de calidad.
    
    \item Un sistema de paralelización eficiente que permite ejecutar experimentos masivos de manera reproducible y recuperable.
    
    \item Optimizaciones algorítmicas (caché de fitness, filtrado de similitud, clasificación adaptativa) que mejoran significativamente la eficiencia computacional.
\end{enumerate}

Estas contribuciones proporcionan una base sólida para futuras investigaciones en optimización multiobjetivo de sistemas de manufactura.
