\chapter{Conclusiones y Trabajo Futuro}

\section{Conclusiones}

Este proyecto de tesis ha logrado establecer una base sólida para la implementación de algoritmos evolutivos multiobjetivo en la optimización de líneas de ensamblaje. A continuación se presentan las conclusiones principales del trabajo realizado:

\begin{itemize}

\item[•] El enfoque multiobjetivo implementado mediante NSGA-II y su variante memética representa una evolución significativa respecto a métodos monoobjetivo tradicionales. La formulación de tres objetivos simultáneos (makespan, balance de carga y consumo energético) permite capturar la complejidad real de los sistemas de manufactura, proporcionando al decisor múltiples soluciones Pareto-óptimas.

\item[•] La implementación del algoritmo NSGA-II con operadores especializados para problemas de programación en líneas de ensamblaje (cruce uniforme por etapa, mutaciones stage-aware) ha demostrado ser efectiva para explorar el espacio de soluciones manteniendo la validez de las configuraciones generadas. Los operadores desarrollados están específicamente adaptados a la estructura de etapas del problema, respetando las restricciones de máquinas disponibles por etapa.

\item[•] La adaptación del problema monoobjetivo original a un enfoque multiobjetivo con tres objetivos permite considerar simultáneamente aspectos de productividad, equidad operativa y sostenibilidad energética. Esta transformación enriquece el análisis y proporciona soluciones más completas para sistemas de manufactura modernos.

\item[•] El desarrollo de la función de fitness multiobjetivo que considera el desgaste de máquinas y procesos de enfriamiento añade realismo al modelo, permitiendo que el algoritmo opere bajo condiciones similares a entornos industriales reales.

\item[•] El proceso sistemático de tunning multiobjetivo, evaluando 1,890 configuraciones con 30 semillas cada una (56,700 ejecuciones totales), permitió identificar la configuración óptima de hiperparámetros que maximiza el balance entre los tres objetivos considerados.

\item[•] La comparación exhaustiva de operadores demostró que la combinación de cruce uniforme y mutación swap proporciona los mejores resultados en términos de calidad del frente de Pareto y robustez estadística.

\item[•] La versión memética del NSGA-II muestra mejoras moderadas pero consistentes (0.11\% en score agregado) sobre la versión estándar, justificando el overhead computacional del 10.22\% para aplicaciones donde la calidad es prioritaria.

\item[•] Las optimizaciones implementadas (caché de fitness, filtrado de similitud, clasificación adaptativa, búsqueda local selectiva) lograron reducir el tiempo de ejecución en aproximadamente 60\%, de ~60 segundos a ~25 segundos por ejecución.

\item[•] El sistema de paralelización de experimentos permite ejecutar experimentos masivos de manera eficiente y reproducible, facilitando la validación estadística de los resultados mediante múltiples semillas independientes.

\end{itemize}

\subsection{Resultados Principales Obtenidos}

Los experimentos sistemáticos realizados han proporcionado evidencia empírica sólida sobre la efectividad del enfoque multiobjetivo propuesto:

\begin{enumerate}
    \item \textbf{Configuración óptima identificada}: Población de 150 individuos, 600 generaciones, probabilidad de cruce 0.7, probabilidad de mutación 0.75, búsqueda local cada 20 generaciones con 5 iteraciones.

    \item \textbf{Mejor combinación de operadores}: Cruce uniforme + Mutación swap, con score agregado de 2.6065 ± 0.0086.
    
    \item \textbf{Superioridad de la versión memética}: Mejora del 0.11\% en score agregado con overhead computacional aceptable del 10.22\%.
    
    \item \textbf{Calidad del frente de Pareto}: 78 soluciones únicas que exploran efectivamente el espacio de compromiso entre los tres objetivos.

    \item \textbf{Robustez estadística}: Desviaciones estándar bajas (0.0086-0.0115) que demuestran consistencia en los resultados.
\end{enumerate}

\section{Trabajo Futuro}

Aunque se han completado exitosamente los experimentos de tunning multiobjetivo y comparación de operadores, se identifican las siguientes líneas de trabajo futuro:

\subsection*{Análisis de Escalabilidad y Performance}
Se recomienda evaluar la escalabilidad del algoritmo implementado con diferentes tamaños de instancias del problema (número de pedidos, número de máquinas, etc.), analizando tanto el tiempo de ejecución como la calidad de las soluciones obtenidas. Este análisis permitirá identificar limitaciones y oportunidades de mejora para instancias más grandes (100+ pedidos).

\subsection*{Extensión a Problemas Dinámicos}
Una línea de investigación prometedora es la extensión del algoritmo a problemas dinámicos donde los pedidos llegan de forma estocástica durante la ejecución. Esto requeriría adaptar el algoritmo para reoptimizar la programación cuando se reciben nuevos pedidos.

\subsection*{Integración con Sistemas de Decisión Multi-Criterio}
El frente de Pareto obtenido puede integrarse con métodos de decisión multi-criterio (como AHP, TOPSIS, o métodos de agregación ponderada) para ayudar al decisor a seleccionar la solución final según sus preferencias específicas.

\subsection*{Validación en Entornos Industriales Reales}
Aunque el modelo considera aspectos realistas (desgaste de máquinas, enfriamiento), la validación en entornos industriales reales proporcionaría evidencia adicional sobre la aplicabilidad práctica del enfoque propuesto.

\subsection*{Optimizaciones Adicionales}
Se pueden explorar optimizaciones adicionales como:
\begin{itemize}
    \item Paralelización del algoritmo mismo (no solo de los experimentos) para reducir el tiempo de ejecución en instancias grandes.
    \item Técnicas de reducción de espacio de búsqueda mediante conocimiento del dominio.
    \item Algoritmos híbridos que combinen NSGA-II con otras metaheurísticas (como simulated annealing o tabu search).
\end{itemize}

%\afterpage{\blankpage}